\documentclass{article}
\usepackage[utf8]{inputenc}
\usepackage{graphicx}
\usepackage{listings}
\graphicspath{ {images/} }
\usepackage{pgf}
\usepackage{pgfpages}

\pgfpagesdeclarelayout{boxed}
{
  \edef\pgfpageoptionborder{0pt}
}
{
  \pgfpagesphysicalpageoptions
  {%
    logical pages=1,%
  }
  \pgfpageslogicalpageoptions{1}
  {
    border code=\pgfsetlinewidth{2pt}\pgfstroke,%
    border shrink=\pgfpageoptionborder,%
    resized width=.95\pgfphysicalwidth,%
    resized height=.95\pgfphysicalheight,%
    center=\pgfpoint{.5\pgfphysicalwidth}{.5\pgfphysicalheight}%
  }%
}

\pgfpagesuselayout{boxed}

\title{In The Name Of God}
\author{Sajjad Saadatpour}
\date{December 2020}

\begin{document}
    \maketitle
    \section*{Introduction}
       This is OSLab HomeWork9 File.
       
    \begin{center}
        \includegraphics[width=50mm]{Linux-logo.png}
    \end{center}
    
    \begin{center}
        \begin{tabular}{ |c|c|c| } 
            \hline
                A11 & A12 & A13 \\ \hline
                A21 & A22 & A23 \\ \hline
                A31 & A32 & A33 \\ 
            \hline
        \end{tabular}
    \end{center}
    
    \[ ax^2 + bx + c = 0 \]
    \[
        \lim_{h \rightarrow 0 } \frac{f(x+h)-f(x)}{h}
    \]
    
    \begin{lstlisting}
        #include "iostream"
       
        using namespace std;
        int main()
        {
            int a[10];
            for(int i = 0; i<10 ; i++)
                cin >>a[i];
        }

    \end{lstlisting}

% \section{Introduction}

\end{document}
